\section{Teory of Helicopters}

\subsection{Equations of Motion for Rigid Airframe}

  As shown in \cite{Cooke}...

  The axes to be used are the helicopter body axes $(O,x,y,z)$ fixed in the helicopter and with its origin at the body axes centre. The components of velocity and force along the $Ox$, $Oy$ and $Oz$ axes are $U$, $V$, $W$ and $X$, $Y$, $Z$, respectively. The components of the rates of rotation about the same axes are $p$, $q$ and $r$ and the moments $L$, $M$ and $N$.

  Considering the position that the position of the centre of gravity $CG$ is given by the co-ordinates $d_x$, $d_y$ and $d_z$ relative to the body axes centre, the absolute velocity of $CG$ is given by $u'$, $v'$ and $w'$:

  \begin{equation}
  	u' = U - r d_y + q d_z \qquad v' = V - p d_z + r d_x \qquad w' = W - q d_x + p d_y
  \end{equation}

  \noindent
  and similarly, for the accelerations of the $CG$:

  \begin{equation}
  	a'_x = \dot{U}' - r v' + q w' \qquad a'_y = \dot{v}' - p w' + r u' \qquad a_z' = \dot{w}' - q u' + p v'
  \end{equation}

\subsection{Relationship between feathering law and flapping law for hovering}
  
  The feathering control law is given by: $$ \theta(\phi) = \theta_0 + \theta_{1C}\cos{\phi} + \theta_{1S}\sin{\phi}$$

  Then, it can be assumed that the flapping angle can be reduced to its first harmonic: $$\beta(\phi) = \beta_0 + \beta_{1C}\cos{\phi} + \beta_{1S}\sin{\phi}$$. Then the constant part $\beta_0$ is given by: $$\beta_0 = \frac{\alpha_\beta \theta_0 - \delta_\beta \lambda_{i0}}{\lambda_\beta^2}$$.

  \begin{eqarray}
    \beta_{1C} = \frac{\alpha_\beta}{(\lambda_\beta^2 - 1)^2 + \eta}
  \end{eqarray}